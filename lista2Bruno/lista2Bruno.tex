\documentclass[12pt]{scrartcl}
\usepackage{graphicx}
\usepackage[brazil]{babel}
\usepackage [utf8] { inputenc }
\usepackage [T1] { fontenc }
\usepackage{minted}
\usepackage{amsmath}
\usepackage{amsfonts,amssymb}
\renewcommand{\thesubsection}{\thesection.\alph{subsection}}
\usepackage{tikz, wasysym}
\usetikzlibrary{automata,positioning}




\begin{document}
\author{Aluno: \textbf{Bruno Kobi Valadares de Amorim}}


\title{Lista de Exercícios 2 \newline\newline
\large Teoria da Computação — Mestrado em Computação Aplicada\newline
\large Prof. Jefferson O. Andrade\newline
}
\date{2022/2}
\maketitle

\section{Exercício}
Seja CountGs um problema computacional que recebe I, uma string ASCII, como entrada.
A solução (única) é o número de vezes que “G” ocorre em I, escrito em notação decimal.

\subsection{}
O CountGs tem alguma instância positiva? Se sim, dê um exemplo.
\newline\newline\textbf{R:}\ Sim, CountGs não tem retorno de "no" em nenhuma combinação de entrada ASCII, então é uma instância positiva. conforme o Livro:"For problems defined on ASCII inputs, an instance is negative if its solution set is the singleton {“no”}. Otherwise the instance is positive."  


\begin{minted}{python}
def CountGs(I):
    total = I.count("G")  
    potencia = 0
    while total >= 10:
        potencia = potencia + 1
        total = total / 10
        print(potencia)
        continue
    return str(total)+'X10^'+str(potencia)
\end{minted}

\subsection{}
O CountGs tem alguma instância negativa? Se sim, dê um exemplo.
\newline\newline\textbf{R:}\ Não


\subsection{}
Sugira uma definição de CountGsDecision, um problema de decisão que intuitivamente parece exigir o mesmo processo computacional que o problema da função
CountGs.

\begin{minted}{python}
def CountGsDecision(I):
    if I.count("G") > 0:
        return 'yes'
    else:
        return 'no'
\end{minted}

\section{Exercício}
Seja L uma linguagem decidível
\subsection{}
Prove que $\bar{L}$ também é decidível.
\newline\newline\textbf{R:}\ \ $\bar{L}$ = \{(B,W) \ | \ B é um AFD que aceita W de entrada\}
\newline Se tanto L quanto $\bar{L}$ são Turing-reconhecíveis, fazemos M1 ser o reconhecedor para L e M2 o reconhecedor para $\bar{L}$.
\newline W = "aa"
\newline M1 = ((q1,q2).(a,b),((q1,a,q2),(q1,b,q2),(q2,a,q2),(q2,b,q2)),q1,(q2))
\newline \textbf{Testando W na máquina de turing M1 = Aceito}
\newline M2 = ((q1,q2).(not a, not b),((q1,not a,q2),(q1,not b,q2),(q2,a,q2),(q2,not b,q2)),q1,(q2))
\newline \textbf{Testando W na máquina de turing M2 = Rejeitado}
\newline\textbf{ Provando que $\bar{L}$ também é decidível. }




\subsection{}
Prove que a união e a interseção de linguagens decidível também são decidível
\newline\newline\textbf{R:} Para provar esse teorema construimos um novo AFC C apartir de A e B , tal que C aceite somente aquelas cadeias são aceitas ou por A ou por B, não por ambos Se A e B reconhecem a mesma linguagem C não aceitará nada.
\newline MtA = ((q1,q2).(a),((q1,a,q2),(q2,a,q2)),q1,(q2))
\newline MtB = ((q1,q2).(b),((q1,b,q2),(q2,b,q2)),q1,(q2))
\newline A$\cup$B = \{(C,W) \ | \ C é um AFD que aceita W não está contindo em A e B ao mesmo tempo\}
\newline \textbf{O W não é aceito na MTc}


\subsection{}
Sendo L decidível, pode-se sempre afirmar que L* é decidível também? Prove esta afirmação ou dê um contraexemplo.
 \newline Defini-se, para uma linguagem L qualquer, o seu fechamento, L*
 \newline Por exemplo:
 \newline Seja a linguagem L = {0, 11}
 \newline  L * = { λ, 0, 11, 00, 011, 110, 1111, 000, 0011, 0110, .... } 
 \newline Se tanto L quanto L* são Turing-reconhecíveis, fazemos M1 ser o reconhecedor para L e M2 o reconhecedor para L*.
 \newline W = "aa"
\newline M1 = ((q1,q2).(a),((q1,a,q2),(q2,a,q2)),q1,(q2)) 
\newline M2 = ((q1,q2).(a,b),((q1,a,q2),(q1,b,q2),(q2,a,q2),(q2,b,q2)),q1,(q2))
\newline \textbf{O W aceito na M1, e também é aceito pelo M2}
\newline\textbf{ Provando que L* também é decidível}




\section{Exercício}
Especifique uma Máquina de Turing (MT) que receba uma sting binária na fita e,
ao final do processamento, duplique esta string. Por exemplo, suponha que a cadeia
de entrada seja “$\circledast$00110”, ao final do processamento a máquina deve deixar a fica com
“$\circledast$0011000110”. O caractere “$\circledast$” está sendo usado com símbolo de início de fita e não
deve ser duplicado.\textbf{(Legenda: $\Box = space,$\  $\circledast = start,$ \  $?=aux$)}

\begin{tikzpicture}[shorten >=1pt,node distance=2cm,on grid,auto]
   \node[state,initial] (0) {$q_0$};
   \node[state] (1) [right=of 0] {$q_1$};
  \node[state] (2)  [right=of 1]{$q_2$};
   \node[state] (11) [below=of 2] {$q_1_1$};
  \node[state] (3) [right=of 11] {$q_3$};
\node[state] (4)  [right=of 3] {$q_4$};
  \node[state] (5)  [right=of 4] {$q_5$};
  \node[state] (6) [right=of 5,yshift=1cm] {$q_6$};
  \node[state] (7)  [above=of 3, yshift=2cm]{$q_7$};
    \node[state] (8) [right=of 7] {$q_8$};
     \node[state] (9) [right=of 8] {$q_9$};
      \node[state] (10) [right=of 9,yshift=-1cm] {$q_1_0$};
       \node[state] (12) [below=of 11] {$q_1_2$};
     \node[state] (13) [below=of 12,yshift=-1cm] {$q_1_3$};
      \node[state] (14) [right=of 13,xshift=2cm] {$q_1_4$};
       \node[state] (15) [left=of 13] {$q_1_5$};
      \node[state] (16) [below=of 13] {$q_1_6$};
      
     
   \node[state,accepting] (17) [right=of 16,xshift=2cm] {$Halt$};
   \path[->]
   
    (0) edge [loop  above]      node {\tiny$\begin{matrix} $0/0, R$ \\ $1/1, R$ \end{matrix}$}
    (0) edge                    node {\tiny$\Box/?,L$}(1)
    (1) edge [loop  above]      node {\tiny$\begin{matrix} $0/0, L$ \\ $1/1, L$ \end{matrix}$}
    (1) edge                    node {\tiny$\Box/\Box,R$}(2)
    (2) edge                    node {\tiny$0/Z,R$}(3)
    (2) edge                    node {\tiny$1/U,R$}(7)
    (2) edge                    node {\tiny$?/?,L$}(11)
    (3) edge [loop  below]      node {\tiny$\begin{matrix} $0/0, R$ \\ $1/1, R$ \end{matrix}$}
    (3) edge                    node {\tiny$?/?,R$}(4)
    (4) edge [loop  below]      node {\tiny$\begin{matrix} $0/0, R$ \\ $1/1, R$ \end{matrix}$}
    (4) edge                    node {\tiny$\Box/0,L$}(5)
    (5) edge [loop  below]      node {\tiny$\begin{matrix} $0/0, L$ \\ $1/1, L$ \end{matrix}$}
    (5) edge                    node {\tiny$?/?,L$}(6)
    (6) edge [loop  right]      node {\tiny$\begin{matrix} $0/0, L$ \\ $1/1, L$ \end{matrix}$}
    (6) edge                    node {\tiny$Z/Z,R$}(2)
    (7) edge [loop  above]      node {\tiny$\begin{matrix} $0/0, R$ \\ $1/1, R$ \end{matrix}$}
    (7) edge                    node {\tiny$?/?,R$}(8)
    (8) edge [loop  above]      node {\tiny$\begin{matrix} $0/0, R$ \\ $1/1, R$ \end{matrix}$}
    (8) edge                    node {\tiny$\Box/1,L$}(9)
    (9) edge [loop  above]      node {\tiny$\begin{matrix} $0/0, L$ \\ $1/1, L$ \end{matrix}$}
    (9) edge                    node {\tiny$?/?,L$}(10)
    (10) edge [loop  right]      node {\tiny$\begin{matrix} $0/0, L$ \\ $1/1, L$ \end{matrix}$}
    (10) edge                    node {\tiny$U/U,R$}(2)
    (11) edge [loop  left]      node {\tiny$\begin{matrix} $Z/0, L$ \\ $U/1, L$ \end{matrix}$}
    (11) edge                    node {\tiny$\Box/\Box,R$}(12)
     (12) edge [loop  left]      node {\tiny$\begin{matrix} $0/0, R$ \\ $1/1, R$ \end{matrix}$}
    (12) edge                    node {\tiny$?/?,R$}(13)
        (13) edge                    node {\tiny$1/?,L$}(14)
         (13) edge                    node {\tiny$0/?,L$}(15)
          (13) edge                    node {\tiny$\Box/\Box,L$}(16)
           (14) edge                    node {\tiny$?/1,R$}(12)
            (15) edge                    node {\tiny$?/0,R$}(12)
             (16) edge                    node {\tiny$\circledast/\circledast,N $}(17)
            (16) edge [loop  left]      node {\tiny$\begin{matrix} $0/0, L$ \\ $1/1, L$ \\ $ ?/\Box, L$  \end{matrix}$}
           
\end{tikzpicture}

\begin{minted}{python}
;Maquina de turing duplica binario
;http://morphett.info/turing/turing.html?a9e27fae74809ef7ce599afff5901fba

;Estado q0.
0 0 0 r 0
0 1 1 r 0
0 _ ? l 1

;Estado q1.
1 0 0 l 1
1 1 1 l 1
1 _ _ r 2

;Estado q2.
2 0 Z r 3
2 1 U r 7
2 ? ? l 11

;Estado q3.
3 0 0 r 3
3 1 1 r 3
3 ? ? r 4

;Estado q4.
4 0 0 r 4
4 1 1 r 4
4 _ 0 l 5

;Estado q5.
5 0 0 l 5
5 1 1 l 5
5 ? ? l 6

;Estado q6.
6 0 0 l 6
6 1 1 l 6
6 Z Z r 2

;Estado q7.
7 0 0 r 7
7 1 1 r 7
7 ? ? r 8

;Estado q8.
8 0 0 r 8
8 1 1 r 8
8 _ 1 l 9

;Estado q9.
9 0 0 l 9
9 1 1 l 9
9 ? ? l 10

;Estado q10.
10 0 0 l 10
10 1 1 l 10
10 U U r 2

;Estado q11.
11 U 1 l 11
11 Z 0 l 11
11 _ _ r 12

;Estado q12.
12 0 0 r 12
12 1 1 r 12
12 ? ? r 13

;Estado q13.
13 1 ? l 14
13 0 ? l 15
13 _ _ l 16

;Estado q14.
14 ? 1 r 12

;Estado q15.
15 ? 0 r 12

;Estado q16.
16 1 1 l 16
16 0 0 l 16
16 ? _ l 16
16 * * * halt-accept

\end{minted}




\section{Exercício}
Especifique uma Máquina de Turing que reverta a sua entrada. Por exemplo, para a
entrada “$\circledast$GATTACA” a saída seria “$\circledast$ACATTAG”. Você pode assumir que a entrada é
uma string genética. O caractere “$\circledast$” está sendo usado com símbolo de início de fita e
não deve ser revertido.
\textbf{(Legenda: $\Box = space,$\  $\circledast = start,$ \  $?=aux$)}  

\begin{tikzpicture}[shorten >=1pt,node distance=4cm,on grid,auto]
   \node[state,initial] (0) {$q_0$};
   \node[state] (1) [right=of 0] {$q_1$};
  \node[state] (2)  [right=of 1]{$q_4$};
  \node[state] (3) [above=of 2] {$q_3$};
\node[state] (4)  [above=of 3] {$q_2$};
  \node[state] (5)  [below=of 2] {$q_5$};
  \node[state] (6) [right=of 5,xshift=2cm] {$q_6$};
  \node[state] (7)  [below=of 1]{$q_7$};
    \node[state] (8) [below=of 7] {$q_8$};
    \node[state,accepting] (9) [right=of 8,xshift=2cm] {$Halt$};
   \path[->]
   
    (0) edge [loop  above]      node {\tiny$\begin{matrix} $A/A, R$ \\ $C/C, R$ \\$G/G, R$ \\$T/T, R$ \end{matrix}$}
    (0) edge                    node {\footnotesize$\Box/\Box,L$}(1)
    (1) edge [loop  above]      node {\footnotesize$?/?,L$}(1)
    (1) edge                    node {\tiny$A/?,R$}(4)
    (1) edge                    node {\tiny$C/?,R$}(3)
    (1) edge                    node {\tiny$G/?,R$}(2)
    (1) edge                    node {\tiny$T/?,R$}(5)
    (1) edge                    node {\footnotesize$\circledast/\Box,R$}(7)
    (2) edge [loop above]       node {\tiny$\begin{matrix} $A/A, R$ \\ $C/C, R$ \\$G/G, R$ \\$T/T, R$\\$?/?, R$   \end{matrix}$} (2)
    (2) edge                     node {\tiny$\Box/G,L$}(6)
    (3) edge [loop above]       node {\tiny$\begin{matrix} $A/A, R$ \\ $C/C, R$ \\$G/G, R$ \\$T/T, R$\\$?/?, R$   \end{matrix}$} (2)
    (3) edge                     node {\tiny$\Box/C,L$}(6)
    (4) edge [loop above]       node {\tiny$\begin{matrix} $A/A, R$ \\ $C/C, R$ \\$G/G, R$ \\$T/T, R$\\$?/?, R$   \end{matrix}$} (2)
    (4) edge                     node {\tiny$\Box/A,L$}(6)
    (5) edge [loop below]       node {\tiny$\begin{matrix} $A/A, R$ \\ $C/C, R$ \\$G/G, R$ \\$T/T, R$\\$?/?, R$   \end{matrix}$} (2)
     (5) edge                     node {\tiny$\Box/T,L$}(6)
    (6) edge [loop below]       node {\tiny$\begin{matrix} $A/A, L$ \\ $C/C, L$ \\$G/G, L$ \\$T/T, L$ \end{matrix}$} (2)
    (6) edge                        node {\tiny$?/?,L$}(1)
    (7) edge [loop right]        node {\tiny$?/\Box,R$}(7)
    (7) edge                       node {\tiny$\begin{matrix} $A/A, L$ \\ $C/C, L$ \\$G/G, L$ \\$T/T, L$ \end{matrix}$} (8)
     (8) edge        node {\footnotesize$\Box/\circledast,N$}(9)
\end{tikzpicture}

\begin{minted}{python}
;Maquina de turing para inverter uma string genética
http://morphett.info/turing/turing.html?aa8aa1d47ae552aab6882ee6d03d63b0 

;Estado q0.
0 A A r 0
0 C C r 0
0 G G r 0
0 T T r 0
0 _ _ l 1


;Estado q1.
1 A ? r 2
1 C ? r 3
1 G ? r 4
1 T ? r 5
1 ? ? l 1
1 * _ r 7

;Estado q2.
2 A A r 2
2 C C r 2
2 G G r 2
2 T T r 2
2 ? ? r 2
2 _ A l 6

;Estado q3.
3 A A r 3
3 C C r 3
3 G G r 3
3 T T r 3
3 ? ? r 3
3 _ C l 6

;Estado q4.
4 A A r 4
4 C C r 4
4 G G r 4
4 T T r 4
4 ? ? r 4
4 _ G l 6

;Estado q5.
5 A A r 5
5 C C r 5
5 G G r 5
5 T T r 5
5 ? ? r 5
5 _ T l 6

;Estado q6.
6 A A l 6
6 C C l 6
6 G G l 6
6 T T l 6
6 ? ? l 1

;Estado q7.
7 A A l 8
7 C C l 8
7 G G l 8
7 T T l 8
7 ? _ r 7

;Estado q8.
8 _ * * halt-accept
\end{minted}




\section{Exercício}
Escreva o programa1 applyBothTwice que receba como entrada uma única string S.
O parâmetro S codifica três strings, P, Q e I utilizando a codificação ESS [1, pag. 61]
da seguinte forma: S = ESS((P, Q), I). Sendo que P e Q são programas. A saída de
applyBothTwice deve ser Q(P(Q(P(I)))).

\begin{minted}{python}
import utils
from utils import rf
from universal import universal

I = 'yes'
S1 = utils.ESS(rf('P.py'), rf('Q.py'))

# String unica de entrada
S = utils.ESS(S1, I)


def applyBothTwice(S):
    (S1, I) = utils.DESS(S)
    (P, Q) = utils.DESS(S1)

    # Q(P(Q(P(I)))) formato de saida desejado
    x1 = universal(P, I)
    x2 = universal(Q, x1)
    x3 = universal(P, x2)
    x4 = universal(Q, x3)

    return x4
\end{minted}






\end{document}


